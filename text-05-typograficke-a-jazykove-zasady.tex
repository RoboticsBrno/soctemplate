\chapter{Typografické a jazykové zásady}
Při tisku odborného textu typu technická zpráva (anglicky technical report), ke kterému patří například i text kvalifikačních prací, se často volí formát A4 a často se tiskne pouze po jedné straně papíru.
V~takovém případě volte levý okraj všech stránek o~něco větší než pravý – v~tomto místě budou papíry svázány a technologie vazby si tento požadavek vynucuje.
Při vazbě s~pevným hřbetem by se levý okraj měl dělat o~něco širší pro tlusté svazky, protože se stránky budou hůře rozevírat a levý okraj se tak bude oku méně odhalovat.

Horní a spodní okraj volte stejně veliký, případně potištěnou část posuňte mírně nahoru (horní okraj menší než dolní).
Počítejte s~tím, že při vazbě budou okraje mírně oříznuty.

Stupeň písma u~nadpisů různé úrovně volíme podle standardních typografických pravidel.
Pro všechny uvedené druhy nadpisů se obvykle používá polotučné nebo tučné písmo (jednotně buď všude polotučné, nebo všude tučné).
Proklad se volí tak, aby se následující text běžných odstavců sázel pokud možno na pevný rejstřík, to znamená jakoby na linky s~předem definovanou a pevnou roztečí.

Uspořádání jednotlivých částí textu musí být přehledné a logické.
Je třeba odlišit názvy kapitol a podkapitol – píšeme je malými písmeny kromě velkých začátečních písmen.
U~jednotlivých odstavců textu odsazujeme první řádek odstavce asi o~jeden až dva čtverčíky (vždy o~stejnou, předem zvolenou hodnotu), tedy přibližně o~dvě šířky velkého písmene M základního textu.
Poslední řádek předchozího odstavce a první řádek následujícího odstavce se v~takovém případě neoddělují svislou mezerou.
Proklad mezi těmito řádky je stejný jako proklad mezi řádky uvnitř odstavce.

Při vkládání obrázků volte jejich rozměry tak, aby nepřesáhly oblast, do které se tiskne text (tj.
okraje textu ze všech stran).
Pro velké obrázky vyčleňte samostatnou stránku.
Obrázky nebo tabulky o~rozměrech větších než A4 umístěte do písemné zprávy formou skládanky všité do přílohy nebo vložené do záložek na zadní desce.

Obrázky i tabulky musí být pořadově očíslovány.
Číslování se volí buď průběžné v~rámci celého textu, nebo - což bývá praktičtější – průběžné v~rámci kapitoly.
V~druhém případě se číslo tabulky nebo obrázku skládá z~čísla kapitoly a čísla obrázku/tabulky v~rámci kapitoly – čísla jsou oddělena tečkou.
Čísla podkapitol nemají na číslování obrázků a tabulek žádný vliv.

Tabulky a obrázky používají své vlastní, nezávislé číselné řady.
Z~toho vyplývá, že v~odkazech uvnitř textu musíme kromě čísla udat i informaci o~tom, zda se jedná o~obrázek či tabulku (například „… viz tabulka 2.7…“).
Dodržování této zásady je ostatně velmi přirozené.

Pro odkazy na stránky, na čísla kapitol a podkapitol, na čísla obrázků a tabulek a v~dalších podobných příkladech využíváme speciálních prostředků DTP programu, které zajistí vygenerování správného čísla i v~případě, že se text posune díky změnám samotného textu nebo díky úpravě parametrů sazby.

Rovnice, na které se budeme v~textu odvolávat, opatříme pořadovými čísly při pravém okraji příslušného řádku.
Tato pořadová čísla se píší v~kulatých závorkách.
Číslování rovnic může být průběžné v~textu nebo v~jednotlivých kapitolách.
Mezeru neděláme tam, kde se spojují číslice s~písmeny v~jedno slovo nebo v~jeden znak – například 25krát.

Členicí (interpunkční) znaménka tečka, čárka, středník, dvojtečka, otazník a vykřičník, jakož i uzavírací závorky a uvozovky se přimykají k~předcházejícímu slovu bez mezery.
Mezera se dělá až za nimi.
To se ovšem netýká desetinné čárky (nebo desetinné tečky).
Otevírací závorka a přední uvozovky se přimykají k~následujícímu slovu a mezera se vynechává před nimi – (takto) a "takto".

Lomítko se píše bez mezer.
Například školní rok 2013/2014.

\section{Co je to normovaná stránka?}
Pojem normovaná stránka se vztahuje k~posuzování objemu práce, nikoliv k~počtu vytištěných listů.
Z~historického hlediska jde o~počet stránek rukopisu, který se psal psacím strojem na speciální předtištěné formuláře při dodržení průměrné délky řádku 60 znaků a při 30 řádcích na stránku rukopisu.
Vzhledem k~zápisu korekturních značek se používalo řádkování 2 (ob jeden řádek).
Tyto údaje (počet znaků na řádek, počet řádků a proklad mezi nimi) se nijak nevztahují ke konečnému vytištěnému výsledku.
Používají se pouze pro posouzení rozsahu.
Jednou normovanou stránkou se tedy rozumí 60*30 = 1800 znaků.
Obrázky zařazené do textu se započítávají do rozsahu písemné práce odhadem jako množství textu, které by ve výsledném dokumentu potisklo stejně velkou plochu.

Orientační rozsah práce v~normostranách lze v~programu Microsoft Word zjistit pomocí funkce \It{Počet slov} v~menu \It{Nástroje}, když hodnotu \It{Znaky (včetně mezer)} vydělíte konstantou 1800.
Do rozsahu práce se započítává pouze text uvedený v~jádru práce.
Části jako abstrakt, klíčová slova, prohlášení, obsah, literatura nebo přílohy se do rozsahu práce nepočítají.
Je proto nutné nejdříve označit jádro práce a teprve pak si nechat spočítat počet znaků.
Přibližný rozsah obrázků odhadnete ručně.